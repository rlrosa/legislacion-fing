\documentclass[spanish,12pt,a4paper,titlepage]{report}
\usepackage[utf8]{inputenc}
\usepackage{graphicx}
\usepackage{subfig}
\usepackage{float}
\usepackage{wrapfig}
\usepackage{multirow}
\usepackage{caption}
\usepackage[spanish]{babel}
\usepackage[dvips]{hyperref}
\usepackage{amssymb}
\usepackage{listings}
\usepackage{epsfig}
\usepackage{amsmath}
\usepackage{array}
\usepackage{enumerate}
\usepackage[table]{xcolor}
\usepackage{multirow}
%\usepackage[Sonny]{fncychap}
\usepackage[Lenny]{fncychap}
%\usepackage[Glenn]{fncychap}
%\usepackage[Conny]{fncychap}
%\usepackage[Rejne]{fncychap}
%\usepackage[Bjarne]{fncychap}
%\usepackage[Bjornstrup]{fncychap}

%\usepackage{subfiles}
%\usepackage{framed}

\usepackage{color}
\newcommand{\highlight}[1]{\colorbox{yellow}{#1}}    %\highlight{this is some highlighted text}


\setlength{\topmargin}{-1.5cm}
\setlength{\textheight}{25cm}
\setlength{\oddsidemargin}{0.3cm} 
\setlength{\textwidth}{15cm}
\setlength{\columnsep}{0cm}

\begin{document}

\begin{itemize}
\item Caracteres que debe reunir el Trabajo para ser objeto del derecho laboral:

$\rightarrow$ \textbf{S}.\textbf{O}.\textbf{L}.\textbf{A}:
\begin{itemize}
\item \textbf{S}ubordinado
\item \textbf{O}neroso
\item \textbf{L}ibre
\item  por cuenta \textbf{A}jena
\end{itemize}


\item Caracteres que reviste la figura del Trabajador como sujeto del derecho laboral

$\rightarrow$ \textbf{P}or \textbf{F}avor \textbf{P}odés \textbf{DA}rme con \textbf{E}l \textbf{D}edo
\begin{itemize}
\item \textbf{P}ersona \textbf{f}ísica
\item \textbf{P}ersonalísimo
\item Actividad bajo \textbf{D}ependencia y por cuenta \textbf{A}jena
\item La \textbf{E}specialización no es obligatoria.
\item No puede haber \textbf{D}iscrimnación.
\end{itemize}

\item Principios del Derecho del Trabajo:

$\rightarrow$ Sos \textit{\textbf{B}ueno}, pero la \textit{\textbf{R}ealidad} es que sos un \textit{\textbf{C.R.I.P.\footnote{como ``creep'', pero mal escrito.}}}.

\begin{itemize}
\item \textbf{Buena} fe.
\item Primacía de la \textbf{realidad}.
\item \textbf{C}ontinuidad.
\item \textbf{R}azonablidad
\item \textbf{I}rrenunciabilidad
\item \textbf{P}rotector
\end{itemize}

\item Reglas que deben observarse para constituir un sindicato:

$\rightarrow$ La regla del Queso, del queso \textbf{D.A.P.E.}

\begin{itemize}
\item No puede haber \textbf{D}iscriminación.
\item Tiene que haber \textbf{A}utonomía.
\item \textbf{E}specialización: El sindicato tiene que tener un objeto.
\item \textbf{P}ureza: Tienen que ser solamente trabajadores.
\end{itemize}

\item Carácteres del derecho laboral en Uruguay

$\rightarrow$ Who's taking care of the kids? \textbf{TH}e \textbf{CR}azy \textbf{PE}do\textbf{FI}le, \textbf{DA}rling!

\begin{itemize}
\item \textbf{T}uitivo
\item \textbf{H}omocéntrico y eminentemente social
\item \textbf{C}ondicionado por lo económico
\item \textbf{R}eciente/moderno
\item \textbf{P}redominantemente colectivo
\item \textbf{E}structurado sobre normas imperativas/de órden público
\item \textbf{F}ragmentario
\item \textbf{I}nternacionalizado
\item \textbf{D}inámico
\item \textbf{A}utónomo
\end{itemize}

\end{itemize}

\end{document}